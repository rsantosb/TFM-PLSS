%!TEX root = ../TFMRocioSantos.tex

\chapter{Introducción}
\label{cap:introduccion}

\chapterquote{Bueno, algunas veces yo he creído hasta \\seis cosas imposibles antes del desayuno}{Lewis Carroll, Alicia en el País de las Maravillas}

Este capítulo comienza con una breve introducción al tema de este proyecto. A continuación se dedica una sección a los objetivos y al plan de trabajo. Por último, se describe la organización de la memoria.


% *****************************************************************************************
\section{Motivación}
% Introducción al tema del TFM.

Muchos problemas de la vida real requieren el uso de datos imprecisos o desconocidos. Su análisis debe implicar la aplicación de principios matemáticos capaces de captar estas características. 
En el campo de la Inteligencia Artificial, la teoría de soft sets proporciona un marco de trabajo apropiado para la toma de decisiones en las situaciones de falta de información.

Pathway Logic es una herramienta diseñada para tratar con sistemas biológicos simbólicos.
Con esta plataforma se han desarrollado numerosos modelos de rutas de señalización celular.
Esta herramienta facilita la comprensión de los sistemas biológicos complejos y la verificación de hipótesis en el diseño de experimentos.

Una de las características del lenguaje Maude es su capacidad de poder expresar de forma natural una amplia gama de aplicaciones, por ejemplo, Pathway Logic.
Las nuevas funcionalidades en el metaintérprete y la gestión de entradas/salidas de las nuevas versiones permiten manejar de forma eficiente un entorno de ejecución dentro de Maude.

El sistema de reescritura estándar de Maude ofrece todos los términos alcanzables del árbol de búsqueda. 
En algunas aplicaciones es deseable elegir únicamente una opción entre todas las posibles.
A partir de esta situación, este proyecto se propone implementar un entorno de ejecución con diferentes estrategias en el sistema de reescritura que permitan escoger el mejor término reescrito.
Teniendo en cuenta que la información disponible puede ser incompleta, se han definido las estrategias mediante la teoría de soft sets.

Con la intención de proporcionar una aplicación con utilidad real, la implementación de este sistema de reescritura se ha desarrollado con los modelos biológicos de Pathway Logic.



% *****************************************************************************************
\section{Objetivos}
% Descripción de los objetivos del trabajo.

El objetivo principal de este proyecto consiste en desarrollar e implementar nuevas variantes de reescrituras basadas en soft sets en el contexto de Pathway Logic.
\smallskip

Las metas intermedias para alcanzar este objetivo se pueden especificar en:
\begin{enumerate}
\item Implementar una aplicación en Maude con sus sistema de comandos propios.
\item Integrar con Pathway Logic.
\item Definir estrategias de reescritura para guiar la reescritura.
\end{enumerate}

Las trabajos realizados se clasifican en los siguientes áreas:
\begin{itemize}
\item Lógica de reescritura.
\item Soft computing, en especial la toma de decisiones bajo incertidumbre con soft sets.
\item Modelización de sistemas biológicos basados en reglas.
\end{itemize}


% *****************************************************************************************
\section{Plan de trabajo}
% Aquí se describe el plan de trabajo a seguir para la consecución de los objetivos descritos en el apartado anterior.

La realización del proyecto se ha apoyado en el plan de trabajo con el director. 
%
Las primeras reuniones se dedicaron a la concreción del tema y objetivos del proyecto.
Después, cada dos semanas hemos tenido reuniones en las que se repasaban las tareas definidas en la reunión anterior.
Por parte del director se realizaban correcciones y se proponían mejoras.
En cada reunión se concretaba el trabajo a realizar durante las dos semanas siguientes.

Aparte de las reuniones ordinarias, las dudas puntuales sobre cualquier aspecto del trabajo se han resuelto por parte del director por correo electrónico.
En todo momento, las respuestas han sido acertadas y las orientaciones valiosas.
\smallskip

Los hitos que se establecieron para alcanzar los objetivos son:
\begin{enumerate}
\item Búsqueda y definición inicial del tema del trabajo.
\item Profundizar en lenguaje Maude.
\item Investigación sobre Pathway Logic y soft sets.
\item Diseño e implementación de un prototipo básico de la aplicación.
\item Ampliación progresiva de las funcionalidades (tanto en los comandos como en las estrategias).
\item Realización de pruebas. 
\item Redacción de la memoria del proyecto.
\end{enumerate}


% *****************************************************************************************
\section{Organización de la memoria}

La memoria está dividida en los siguientes capítulos:

\begin{itemize}
\item \textbf{Capítulo~\ref{cap:introduccion}}: Introducción. 
Se introduce el tema del TFM, se describen los objetivos del trabajo y se detalla el plan de trabajo seguido para la consecución de los objetivos.

\item \textbf{Capítulo~\ref{cap:estadoDeLaCuestion}}: Estado de la Cuestión. 
En este capítulo se introducen las áreas en las que se fundamenta el trabajo: el lenguaje de programación Maude, el metaintérprete de Maude, la teoría de soft sets y Pathway Logic.

\item \textbf{Capítulo~\ref{cap:descripcionTrabajo}}: Descripción del Trabajo.
En este capítulo se describe el trabajo realizado en el proyecto \textbf{PLSS}: \textbf{P}athway \textbf{L}ogic con \textbf{S}oft \textbf{S}ets.

\item \textbf{Capítulo~\ref{cap:analisis}}: Análisis de Resultados.
En este capítulo se incluyen algunos comentarios sobre el rendimiento de la aplicación y se realiza un análisis comparativo de los resultados obtenidos con los del uso estándar de la reescritura.

\item \textbf{Capítulo~\ref{cap:conclusiones}}: Conclusiones y Trabajo Futuro.
Por último, se han expuesto las conclusiones y las líneas para un trabajo futuro.
\end{itemize}
\medskip

El código elaborado para este proyecto está disponible en el repositorio de GitHub
\url{https://github.com/rsantosb/TFM-PLSS}, bajo la licencia MIT (\url{https://opensource.org/licenses/MIT}).

