%!TEX root = ../TFMRocioSantos.tex

\chapter{Introduction}
\label{cap:introduction}

%Introduction to the subject area. This chapter contains the translation of Chapter \ref{cap:introduccion}.

\chapterquote{Why, sometimes I've believed as many as \\six impossible things before breakfast}{Lewis Carroll, Alice in Wonderland}


This chapter begins with a brief introduction to the topic of this project. It is followed by a section on the objectives and work plan. Finally, the organization of the dissertation is described.


% *****************************************************************************************
\section{Motivation}
% Introducción al tema del TFM.

Many real-life problems require the use of inaccurate or unknown data. Their analysis must involve the application of mathematical principles capable of capturing these characteristics. 
In the field of Artificial Intelligence, the theory of soft sets provides an appropriate framework for decision making in situations of lack of information.

Pathway Logic is a tool designed to deal with symbolic biological systems.
Numerous cellular signaling pathway models have been developed with this platform.
This tool facilitates the understanding of complex biological systems and the verification of hypotheses in the design of experiments.

One of the characteristics of the Maude language is its ability to naturally express a wide range of applications, for example, Pathway Logic.
The new features in the meta-interpreter and input/output management of the new versions allow for efficient management of a runtime environment within Maude.

Maude's standard rewrite system provides all search tree terms that are attainable. 
In some applications, it is desirable to choose only one option from all the possible ones.
Based on this situation, this project aims to implement an execution environment with different strategies in the rewriting system that allows choosing the best rewritten term.
Taking into account that the available information may be incomplete, the strategies have been defined through the theory of soft sets.

With the intention of providing an application with real utility, the implementation of this rewriting system has been developed with Pathway Logic's biological models.



% *****************************************************************************************
\section{Objectives}
% Descripción de los objetivos del trabajo.

The main objective of this project is to develop and implement new variants of rewrites based on soft sets in the context of Pathway Logic.
\smallskip

Intermediate goals for achieving this objective can be specified at:
\begin{enumerate}
\item Implement an application in Maude with its own command system.
\item Integrate with Pathway Logic.
\item Define rewriting strategies to guide rewriting.
\end{enumerate}

The work done is classified into the following areas:
\begin{itemize}
\item Rewriting logic.
\item Soft computing, especially decision making under uncertainty with soft sets.
\item Modeling of biological systems based on rules.
\end{itemize}


% *****************************************************************************************
\section{Work plan}
% Aquí se describe el plan de trabajo a seguir para la consecución de los objetivos descritos en el apartado anterior.

The realization of the project has been supported by the work plan with the director.
%
The first meetings were dedicated to the concretion of the theme and objectives of the project.
Afterwards, every two weeks we had meetings in which the tasks defined in the previous meeting were reviewed.
On the part of the director, corrections were made and improvements proposed.
In each meeting the work to be done during the following two weeks was specified.

Apart from the regular meetings, occasional doubts about any aspect of the work have been resolved by the director by e-mail.
At all times, the answers have been accurate and the guidance valuable.
\smallskip

The milestones that were established to achieve the objectives are:
\begin{enumerate}
\item Search and initial definition of the subject of the work.
\item Going deeper into the Maude language.
\item Research on Pathway Logic and soft sets.
\item Design and implementation of a basic prototype of the application.
\item Progressive extension of the functionalities (both in the commands and in the strategies).
\item Testing. 
\item Drafting of the project report.
\end{enumerate}


% *****************************************************************************************
\section{Organization of the dissertation}

The dissertation is divided into the following chapters:

\begin{itemize}
\item \textbf{Chapter~\ref{cap:introduccion}}: Introduction. 
The theme of the TFM is introduced, the objectives of the work are described and the work plan followed to achieve the objectives is detailed. 

\item \textbf{Chapter~\ref{cap:estadoDeLaCuestion}}: State of the art. 
This chapter introduces the areas on which the work is based: the Maude programming language, the Maude meta-interpreter, soft set theory and Pathway Logic.

\item \textbf{Chapter~\ref{cap:descripcionTrabajo}}: Description of the work.
This chapter describes the work done in the project \textbf{PLSS}: \textbf{P}athway \textbf{L}ogic with \textbf{S}oft \textbf{S}ets.

\item \textbf{Chapter~\ref{cap:analisis}}: Results Analysis.
This chapter includes some comments on the performance of the application and a comparative analysis of the results obtained with those of the standard use of rewriting.

\item \textbf{Chapter~\ref{cap:conclusiones}}: Conclusions and Future Work.
Finally, the conclusions and lines for future work have been set out.
\end{itemize}
\medskip


The code developed for this project is available in the GitHub repository
\url{https://github.com/rsantosb/TFM-PLSS}, under the MIT license (\url{https://opensource.org/licenses/MIT}).



